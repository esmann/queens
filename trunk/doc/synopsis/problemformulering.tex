\subsection*{Problemformulering}
Opgavens formål er at implementere en parallel udgave af Takakens algoritme til løsning af n-dronning-problemet. Algoritmen skal køre på MiG-systemet (Minimum Intrusion Grid) og MiG's one-click arkitektur skal kunne udnyttes til at skaffe ressourcer til beregning af problemet. Formålet er på langt sigt at få beregnet en løsning til N-dronning-problemet for $n=26$, men opgaven er kun at gøre dette muligt ved hjælp af MiG. N-dronning-problemet er et klassisk beregningsproblem, der går ud på at finde antallet af mulige måder n dronninger kan placeres på et "skakbræt" med n x n felter, uden at nogen af dem er istand til at slå hinanden i næste træk. Problemets størrelse stiger eksponentielt med n, og er uhyre beregningstungt for store n, hvorfor store distribuerede systemer ofte benyttes. Hidtil er der kun fundet løsninger for $n \in \{1,...,25\}$. For distlab-gruppen her på diku ville en løsning for $n=26$, beregnet på et MiG-grid, kunne skabe opmærksomhed omkring MiG-systemet. 

MiG er beskrevet indgående i \cite{simplemig} og \cite{mig}, \cite{etsi} beskriver N-dronning-problemet grundigere end ovenstående og præsenterer en løsning for n=25. Appendix queens.c i \cite{etsi} er en udskrift af Takakens algoritme implementeret i C.
One-click muliggør deltagelse i et MiG-grid uden andre forudsætninger end en webbrowser og java. Tilgengæld er denne metode begrænset til at afvikle programmer, der er tilgængelige som java-bytecode. Brug af one-click giver adgang til et enormt (potentielt) antal beregningsressourcer, hvilket er grunden til at benytte one-click i denne opgave.    

Opgaven indeholder altså følgende delproblemer: 
\begin{itemize}
\item At finde en effektiv strategi til parallelisering af Takakens algoritme. Herunder overvejelser omkring den optimale størrelse på delproblemer.
\item Implementation af algoritmen i java på en sådan måde at den kan afvikles af one-click-klienter. 
\item Strategi for indsamling, behandling og præsentation af delresultater. 
\item Første opgave er naturligvis at få et bedre kendskab til MiG.
\end{itemize}
