\documentclass{beamer}

\usepackage{beamerthemesplit}
\usepackage[utf8]{inputenc}

\title{N-dronninge problemet på MiG}
\author{Alex Esmann, Frej Soya, Thomas Clement Mogensen}
\date{\today}

\begin{document}

\frame{\titlepage}
\section{Introduktion}
\frame
{
	\frametitle{Introduktion}

	\begin{itemize}
	\item<1-> Opgaven
	\item<.-> MiG
	\item<.-> One-Click
	\item<.-> NQueen
	\end{itemize}
}


\section{Rapportopsamling}
\frame
{
  \frametitle{Rapportopsamling}

  \begin{itemize}
  \item<1-> Billedetekst, figur 9
  \item<.-> Implementering af Checkpoint 
  \item<.-> Tvungen nedarvning fra Job-klassen
  \end{itemize}
}

\section{Status}
\frame
{
  \frametitle{Status}

  \begin{itemize}
  \item<1-> Hvad er implementeret.
  \item<.-> Hvad er status for den konkrete beregning
  \item<.-> Hvilke problemer er vi stødt på
  \item<.-> Alternativ implementation.
  \end{itemize}
}

\section{Ideer til forbedringer}
\frame
{
  \frametitle{Ideer til forbedringer}

  \begin{itemize}
  \item<1-> Checkpointing kunne returnere efter serialisering
  \item<.-> Schedulering i BOINC, kan vi overføre noget til MiG.
  \item<.-> Opgavestørrelser, problemet opstår først til sidst og der er der andre projekter der vil overtage.      
	\item<.-> Estimat af hvor lang tid den største/mindste opgave tager. 
  \end{itemize}
}
\end{document}
