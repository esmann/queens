\documentclass{beamer}

\usepackage{beamerthemesplit}
\usepackage[utf8]{inputenc}

\title{N-dronninge problemet på MiG}
\author{Alex Esmann, Frej Soya, Thomas Clement Mogensen}
\date{\today}

\begin{document}

\frame{\titlepage}
\frame{\tableofcontents}
\section{Introduktion}
\frame
{
	\frametitle{Introduktion}

	\begin{itemize}
	\item<1-> Opgaven.
	\item<.-> MiG.
	\item<.-> One-Click.
	\item<.-> NQueen.
	\end{itemize}
}


\section{Ting vi mangler at forklare i rapporten}
\frame
{
  \frametitle{Ting vi mangler at forklare i rapporten}

  \begin{itemize}
  \item<1-> De værste fejl.
  \item<.-> Mangler i forhold til problemformuleringen.
  \item<.-> Hvorfor er vi ikke gået videre med løsningen med intelligente	ressourcer.      
	\item<.-> Forklaring af checkpointing mekanismen.
	\item<.-> Ting vi gerne ville have haft tid til.
  \end{itemize}
}

\section{Status}
\frame
{
  \frametitle{Status}

  \begin{itemize}
  \item<1-> Hvad kører/Fejler.
  \item<.-> Benchmarks for stak/hægtet-liste baseret implementation.
  \item<.-> Problemer med MiG.      
  \end{itemize}
}

\section{Ideer til forbedringer}
\frame
{
  \frametitle{Ideer til forbedringer}

  \begin{itemize}
  \item<1-> Checkpointing kunne returnere efter serialisering
  \item<.-> Schedulering i BOINC, kan vi overføre noget til MiG.
  \item<.-> Opgavestørrelser, problemet opstår først til sidst og der er der andre projekter der vil overtage.      
	\item<.-> Estimat af hvor lang tid den største/mindste opgave tager. 
  \end{itemize}
}
\end{document}
