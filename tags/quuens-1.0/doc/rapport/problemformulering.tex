Opgavens form�l er at implementere en parallel udgave af Takakens algoritme til l�sning af n-dronning-problemet. Algoritmen skal k�re p� MiG-systemet (Minimum Intrusion Grid) og MiG's one-click arkitektur skal kunne udnyttes til at skaffe ressourcer til beregning af problemet. Form�let er p� langt sigt at f� beregnet en l�sning til N-dronning-problemet for $n=26$, men opgaven er kun at g�re dette muligt ved hj�lp af MiG. N-dronning-problemet er et klassisk beregningsproblem, der g�r ud p� at finde antallet af mulige m�der n dronninger kan placeres p� et "skakbr�t" med n x n felter, uden at nogen af dem er istand til at sl� hinanden i n�ste tr�k. Problemets st�rrelse stiger eksponentielt med n, og er uhyre beregningstungt for store n, hvorfor store distribuerede systemer ofte benyttes. Hidtil er der kun fundet l�sninger for $n \in \{1,...,25\}$. For distlab-gruppen her p� diku ville en l�sning for $n=26$, beregnet p� et MiG-grid, kunne skabe opm�rksomhed omkring MiG-systemet. 

MiG er beskrevet indg�ende i \cite{simplemig} og \cite{mig}, \cite{etsi} beskriver N-dronning-problemet grundigere end ovenst�ende og pr�senterer en l�sning for n=25. Appendix queens.c i \cite{etsi} er en udskrift af Takakens algoritme implementeret i C.
One-click muligg�r deltagelse i et MiG-grid uden andre foruds�tninger end en webbrowser og java. Tilgeng�ld er denne metode begr�nset til at afvikle programmer, der er tilg�ngelige som java-bytecode. Brug af one-click giver adgang til et enormt (potentielt) antal beregningsressourcer, hvilket er grunden til at benytte one-click i denne opgave.    

Opgaven indeholder alts� f�lgende delproblemer: 
\begin{itemize}
\item At finde en effektiv strategi til parallelisering af Takakens algoritme. Herunder overvejelser omkring den optimale st�rrelse p� delproblemer.
\item Implementation af algoritmen i java p� en s�dan m�de at den kan afvikles af one-click-klienter. 
\item Strategi for indsamling, behandling og pr�sentation af delresultater. 
\item F�rste opgave er naturligvis at f� et bedre kendskab til MiG.
\end{itemize}
