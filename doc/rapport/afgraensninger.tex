Projektets form�l er ikke at beregne en l�sning til N-dronning-problemet for $n=26$, hvilket problemets beregningsm�ssige omfang kombineret med tids- og ressourcebegr�sninger udelukker i praksis. Men kun at muligg�re og forh�bentlig igangs�tte denne beregning. 
Vi vil ikke tage stilling til den benyttede algoritmes korrekthed eller effektivitet, men kun til den bedst mulige strategi for parallelisering. Partitionering af problemdata skal foreg� p� en fornuftig m�de, med tanke p� hvordan det forventes beregningsressourcerne opf�rer sig, men en decideret statisk unders�gelse af midlertidige MiG-ressourcers opf�rsel eller levetid vil ikke blive foretaget\footnote{Med ressourcers opf�rsel t�nkes p� den tid man kan forvente en bruger vil lade sin one-click-klient k�re}. Fordele og ulemper ved MiG eller One-click i forhold til andre grid-systemer falder ogs� udenfor opgavens omfang. 

Ved at at l�se problemet for et lavere $n$ vil vi kunne estimere tiden og/eller antal CPU'er der skal bruges for at l�se for n=26.


