\subsubsection{Generelle implementeringsovervejelser}

\begin{itemize}
\item{Vi er n�dt til at tage h�jde for at l�sningen (antallet af l�singer) vil overstige hvad vi kan repr�sentere som en 32bit v�rdi. }
\end{itemize}

I forbindelse med joboprettelse og resultatindsamling skal vi tr�ffe nogle valg mht.  
\begin{itemize}
\item{Joboprettelsesstrategi - hvad skal der til for at beskrive et job.}
\item{}
\end{itemize}
 
Det skal overvejes om en den rekursive implementation Takaken bruger vil fungere i en one-click-implementation. Skal vi g�re brug af one-click's checkpointing er en rekursiv implementation ikke oplagt, idet jobbets stak ikke bevares p� tv�rs af den serialisation der sker under checkpointing. Det er derfor n�dvendigt at lave en ikke ikke-rekursiv eller hale-rekursiv implementation hvis vi vil undg� selv at skulle gemme og reetablere vores stak ved i forbindelse med checkpointing og genetablering af job fra checkpoint.


\subsubsection{Konkrete implementeringer}




