
Begrundelse
          o N-dronning problemet er velkendt som et meget krævende beregningsproblem, at finde en løsning for n=26 vil være en god måde at promovere MiG-systemet på. Takakens algoritme vil forhåbentlig lette denne opgave. 
    * Arbejdsopgaver (med forventet tidsforbrug for hver opgave)
          o Bekendtgøre os med MiG-systemet og one-click-arkitekturen
          o Skrive en rapport
                + Analyse af algoritmen
                + Samme med mig?
          o Porte koden
          o parallelisere koden
          o Teste koden
          o Rette rapporten igennem
          o Vinter nævnte noget med 1300 cpuer. Hvordan skal vi finde 1300 folk der gider åbne en applet i baggrunden? Er det meningen? (nej, det er ikke meningen, det skal der skrives i afgrænsningen at vi ikke gør)
          

    * Evt. metoder, information og informationskilder (litteratur) eller andet der kan begrunde og uddybe listen over opgaver


http://www.diku.dk/undervisning/2006-2007/2006-2007_b3_dat2b/opskrift-proj-plan.pdf

